% !TeX root = ../falcon.tex


\section{Encoding Formats} \label{sec:spec:encode}

\subsection{Bits and Bytes}

A \emph{byte} is a sequence of eight bits (formally, an \emph{octet}).
Within a byte, bits are ordered from left to right. A byte has a
numerical value, which is obtained by adding the weighted bits; the
leftmost bit, also called ``top bit'' or ``most significant'', has
weight 128; the next bit has weight 64, and so on, until the rightmost
bit, which has weight 1.

Some of the encoding formats defined below use sequences of bits. When
a sequence of bits is represented as bytes, the following rules apply:
\begin{itemize}
\item The first byte will contain the first eight bits of the sequence;
the second byte will contain the next eight bits, and so on.
\item Within each byte, bits are ordered left-to-right in the same order
as they appear in the source bit sequence.
\item If the bit sequence length is not a multiple of 8, up to 7 extra
padding bits are added at the end of the sequence. The extra padding bits
MUST have value zero.
\end{itemize}

This handling of bits matches widely deployed standard, e.g. bit
ordering in the SHA-2 and SHA-3 functions, and \texttt{BIT STRING}
values in ASN.1.

\subsection{Compressing Gaussians} \label{sec:spec:encode:compress}

% TODO: this is not what the code currently implements. This encoding
% process should be tried and benchmarked. Notably, the cut-off at 7
% bits may depend on the value of q.

In \falcon, the signature of a message essentially consists of a polynomial $s \in \bZ_q[x]/(\phi)$ which coefficients are distributed around $0$ according to a discrete Gaussian distribution of standard deviation $\sigma \approx 1.55 \sqrt{q} \ll q$. A naive encoding of $s$ would require about $\lceil \log_2 q\rceil \cdot \deg(\phi)$ bits, which is far from optimal for communication complexity.

In this section we specify algorithms for compressing and decompressing efficiently polynomials such as $s$. The description of this compression procedure is simple:
\begin{enumerate}
 \item
 For each coefficient $s_i$, a compressed string $\str_i$ is defined as follows:
 \begin{enumerate}
  \item The first bit of $\str_i$ is the sign of $s_i$;
  \item The $7$ next bits of $\str_i$ are the $7$ least significant bits of $|s_i|$, in order of significance, i.e. most to least significant;
  \item The last bits of $\str_i$  are an encoding of the most significant bits of $|s_i|$ using unary coding. If $\lfloor|s_i|/2^7\rfloor = k$, then its encoding is $\underbrace{\tt{0\dots 0}}_{k \text{ times}} \tt 1$, which we also note ${{\tt 0}^k \tt 1}$; 
 \end{enumerate}
  \item 
 The compression of $s$ is the concatenated string $\str \gets (\str_0\|\str_1\|\dots\|\str_{n-1})$.
 
 \item 
 \str is padded with zeroes to a fixed width \slen.
\end{enumerate}
This encoding is based on two observations. First, since $s_i \bmod 2^7$ is close to uniform, it is pointless to compress the $7$ least significant bits of $s_i$. Second, if a Huffman table is computed for the most significant bits of $|s_i|$, it results in the unary code we just described. So our unary code is actually a Huffman code for the distribution of the most significant bits of $|s_i|$. A formal description is given in \longcompress.

\begin{algorithm}[!htp]
\caption{$\compress(s, \slen)$ \hfill}\label{alg:compress}
 \begin{algorithmic}[1]
 \Require {A polynomial $s = \sum s_i x^i \in \bZ[x]$ of degree $<n$, a string bitlength $\slen$}
 \Ensure {A compressed representation \str of $s$ of $\slen$ bits, or $\bot$}
 \State{$\str \gets \{\}$}\Comment{\str is the empty string}
 \For{$i$ from $0$ to $n-1$}\Comment{At each step, $\str \gets (\str\|\str_i)$, where $\str_i$ encodes $s_i$}
 \State{$\str \gets (\str\|b)$, where $b = {\tt 1}$ if $s_i < 0$, $b = {\tt 0}$ otherwise} \Comment{Encode the sign of $s_i$}
% \State{$\str \gets (\str\|b_6 b_5 \dots b_0)$, where $b_j = \lfloor|s_i| / 2^j\rfloor \bmod 2 $}
 \State{$\str \gets (\str\|b_6 b_5 \dots b_0)$, where $b_j = (|s_i| \rightshift j)\ \texttt{\&} \ {\tt 0x1}$}\label{line:cbin}
 \Comment{Encode in binary the low bits of $|s_i|$}
% \State{$k \gets \lfloor|s_i| / 2^7\rfloor$}
 \State{$k \gets |s_i| \rightshift 7$}\label{line:k}
 \State{$\str \gets (\str\|{\tt 0}^k \tt 1)$}\label{line:unary} \Comment{Encode in unary the high bits of $|s_i|$}
 \EndFor
 \If{$(|\str| > \slen)$}\label{line:slen}
 \State{$\str \gets \bot$} \Comment{Abort if $\str$ is too long}\label{line:slenbot}
 \Else\label{line:else}
 \State{$\str \gets (\str\|{\tt 0}^{\slen - |\str|})$} \Comment{Pad $\str$ to $\slen$ bits}\label{line:pad}
 \EndIf
 \Return{\str}
 \end{algorithmic}
\end{algorithm}

The corresponding decompression algorithm is given in \longdecompress. For any polynomial $s \in \bZ[x]$ such that $\compress(s, \slen) \neq \bot$, it holds that $\decompress ( \compress(s, \slen), \slen) = s$. We now enforce unique encodings: a polynomial $s$ should have at most one valid encoding $\str$. This is done via three additional checks in \decompress:
\begin{enumerate}
	\item only accept bitstrings of length $\slen = 8\cdot \sigbytelen - 328$ (see \cref{line:fix,line:fix2});
	\item only accept ${\tt 000000001}$ -- and not ${\tt 100000001}$ -- as a valid encoding of the coefficient $0$ (see \cref{line:zero,line:zero2});
	\item force the last bits of $\str$ to be $\tt 0$ (see \cref{line:trail,line:trail2}).
\end{enumerate}


\begin{algorithm}[!htp]
\caption{$\decompress(\str, \slen)$ \hfill}\label{alg:decompress}
	\begin{algorithmic}[1]
	\Ensure {A bitstring $\str = (\str[i])_{i=0, \dots, \slen-1}$, a bitlength $\slen$}
	\Require {A polynomial $s = \sum s_i x^i \in \bZ[x]$, or $\bot$}
	\If{$|\str| \neq \slen$}\label{line:fix} \Comment{Enforce fixed bitlength}
	\Return{$\bot$} \label{line:fix2}
	\EndIf
%	\State{$j \gets 0$}
	\For{$i$ from $0$ to $(n-1)$}
	\State{$s_i' \gets \sum_{j=0}^6 2^{6 - j} \cdot \str[1+j]$}\label{line:dbin} \Comment{We recover the lowest bits of $|s_i|$.}
	\State{$k \gets 0$}\Comment{We recover the highest bits of $|s_i|$.}
	\While{$\str[8+k] = 0$}\label{line:8}
	\State{$k \gets k+1$}
	\EndWhile
	\State{$s_i \gets (-1)^{\str[0]} \cdot (s_i' + 2^7 k)$}\Comment{We recompute $s_i$.}
	\If{($s_i = 0$) and ($\str[0] = \tt 1$)}\label{line:zero} \Comment{Enforce unique encoding if $s_i = 0$}
	\Return{$\bot$}\label{line:zero2}
	\EndIf
	\State{$\str \gets \str[9+k \dots \ell-1]$}\Comment{We remove the bits of \str that encore $s_i$.}
	\EndFor
	\If{$(\str \neq {\tt 0}^{|\str|})$}\label{line:trail} \Comment{Enforce trailing bits at ${\tt 0}$}
	\Return{$\bot$}\label{line:trail2} 
	\EndIf
	\Return {$s = \sum_{i=0}^{n-1} s_i x^i$}
	\end{algorithmic}
\end{algorithm}

\subsection{Signatures}

A \falcon signature consists of two strings \salt and \comps. They may
conceptually be encoded separately, because the salt \salt must be known
\emph{before} beginning to hash the message itself, while the \comps
value can be obtained or verified only after the whole message has been
processed. In a format that supports streamed processing of long
messages, the salt \salt would normally be encoded before the message,
while the \comps value would appear after the message bytes. However,
we here define an encoding that includes both \salt and \comps.

The first byte is a header with the following format (bits indicated
from most to least significant):
\begin{verbatim}
    0 c c 1 n n n n
\end{verbatim}
with these conventions:
\begin{itemize}

    \item The leftmost bit is \texttt{0}, and the fourth bit from the left
    is \texttt{1} (in previous versions of \falcon, these bits may had
    have different values).

    \item Bits \texttt{cc} are \texttt{01} or \texttt{10} to specify the
    encoding method for \comps. Encoding \texttt{01} uses the compression
    algorithm described in Section~\ref{sec:spec:encode:compress};
    encoding \texttt{10} is alternate uncompressed encoding in which each
    coefficient of \comps is encoded over a fixed number of bits.

    \item Bits \texttt{nnnn} encode a value $\ell$ such that the
    \falcon degree is $n = 2^\ell$. $\ell$ must be in the allowed
    range (1 to 10).

\end{itemize}

Following the header byte are the nonce string \salt (40 bytes), then
the encoding of \comps itself.

Signatures are then normally padded with zeros up to the prescribed
length (\sigbytelen). Verifiers may also support unpadded signatures,
which do not have a fixed size, but are (on average) slightly shorter
than padded signatures. \emph{Partial} padding is not valid: if the
signature has padding bytes, then all padding bytes must be zero, and
the total padded length must be equal to \sigbytelen.

When using the alternate uncompressed format (\texttt{cc} is \texttt{10}
in the header byte), all elements of \comps are encoded over exactly 12
bits each (signed big-endian encoding, using two's complement for
negative integers; the valid range is $-2047$ to $+2047$, the value
$-2048$ being forbidden)\footnote{In some reduced versions of \falcon,
with degree 16 or less, fewer bits may be used. These reduced versions
do not offer any security and are used only for research and tests.}.
This uncompressed format yields larger signatures and is meant to
support the uncommon situations in which signature values and signed
messages are secret: uncompressed signatures can be decoded and encoded
with constant-time implementations that do not leak information through
timing-based side channels.

\subsection{Public Keys}

A \falcon public key is a polynomial $h$ whose coefficients are considered
modulo $q$. An encoded public key starts with a header byte:
\begin{verbatim}
    0 0 0 0 n n n n
\end{verbatim}
with these conventions:
\begin{itemize}
\item The four leftmost bits are 0 (in some previous versions of \falcon,
the leftmost bit could have been non-zero).
\item Bits \texttt{nnnn} encode a value $\ell$ such that the degree is
$n = 2^\ell$. $\ell$ must be in the allowed range ($1$ to $10$).
\end{itemize}

After the header byte comes the encoding of $h$: each value (in the $0$
to $q-1$ range) is encoded as a 14-bit sequence (since $q = 12289$, 14 bits
per value are used). The encoded values are concatenated
into a bit sequence of $14n$ bits, which is then represented as
$\lceil 14n/8\rceil$ bytes.

\subsection{Private Keys}

Private keys use the following header byte:
\begin{verbatim}
    0 1 0 1 n n n n
\end{verbatim}
with these conventions:
\begin{itemize}
\item The four leftmost bits are \texttt{0101}.
\item Bits \texttt{nnnn} encode the value $\ell$ such that the degree
is $n = 2^\ell$. $\ell$ must be in the allowed range ($1$ to $10$).
\end{itemize}

Following the header byte are the encodings of $f$, $g$, and $F$,
in that order. Each coordinate is encoded over a fixed number of bits,
which depends on the degree:
\begin{itemize}

    \item Coefficients of $f$ and $g$ use:
    \begin{itemize}
        \item 8 bits each for degrees 2 to 32;
        \item 7 bits each for degrees 64 and 128;
        \item 6 bits each for degrees 256 and 512;
        \item 5 bits each for degree 1024.
    \end{itemize}

    \item Coefficients of $F$ use 8 bits each, regardless of the
    degree.

\end{itemize}
Of course, small degrees do not offer real security, and are meant only
for test and research purposes. In practical situations, the degree
should be 512 or 1024.

Each coefficient uses signed encoding, with two's complement for
negative values. Moreover, the minimal value is forbidden; e.g. when
using degree 512, the valid range for a coefficient of $f$ or $g$
is $-31$ to $+31$; $-32$ is not allowed.

The polynomial $G$ is not encoded. It is recomputed when the key is
loaded, thanks to the NTRU equation:
\begin{equation}
  G = (q + gF)/f \mod \phi
\end{equation}
Since the coefficients of $f$, $g$, $F$ and $G$ are small, this
computation can be done modulo $q$ as well, using the same techniques as
signature verification (e.g. the NTT).

\subsection{NIST API}

The API to be implemented by candidates to the NIST call for
post-quantum algorithms mandates a different convention, in which the
signed message and the signature are packed into a single aggregate
format. In this API, the following encoding is used:
\begin{itemize}

    \item First two bytes are the ``signature length'' (big-endian
    encoding).

    \item Then follows the nonce \salt (40 bytes).

    \item The message data itself appears immediately after the nonce.

    \item The signature comes last. This signature uses a nonce-less
    format:
    \begin{itemize}
        \item Header byte is: \texttt{0010nnnn}
        \item Encoded \comps immediately follows, using compressed
        encoding.
    \end{itemize}

\end{itemize}
There is no signature padding; the signature has a variable length. The
length specified in the first two bytes of the package is the length,
in bytes, of the signature, including its header byte, but not including
the nonce length.
